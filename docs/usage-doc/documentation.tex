\documentclass[titlepage]{article}
% \usepackage{fullpage}
% \usepackage{letterpaper, landscape, margin=2in}
\usepackage{color,soul}
\usepackage{graphicx, graphics}
\usepackage{enumitem}
\usepackage{enumerate}

\begin{document}

\title{\huge STL Script Documentation\\[1ex] \LARGE A domain-specifc programming language to specify and evaluate Signal Temporal Logic formulae.}
\author{
{\Large\textbf{Contributor(s)}}\\[3ex]
% Member 1\\[1ex] Member 2\\[1ex] Member 3\\[1ex] Member 4}
\Large Simon Chu\\[1ex]}
% \date{\today} % make the date as today's date
\date{December 1, 2020}

\maketitle


\section{Motivation}

\noindent{Signal temporal logic (STL) is a temporal logic formalism for specifying properties of continuous signals. It is widely used for analyzing programs in cyber-physical systems (CPS) that interact with physical entities exhibiting continous dynamics.}
\medskip

\noindent{However, neither the industry nor academia has agreed upon a common
standard of STL, and the non-standardized technique has bring about the
discrepancies in analyzing system properties, and resulting in the industry and
academia spending a tremendous amount of effort implementing ad hoc/specialized
APIs that are tailored to specific use cases, cannot communicate with each other,
requires steep learning curve for new personnel that just got involved in the
project, and causing inefficiencies when trying to apply the same reasoning
techniques across projects.}
\medskip

\noindent{A programming language is a bicycle for the mind. By standardizing STL
using a programming language, it effectively support collaboration across
projects and disciplines, effectively reduce the overhead for training new
personnel, and developing ad hoc tools for each project, which allows developer
to focus on reasoning about the STL specifications instead of its implementation
and verification.}
\medskip

\noindent{I will introduce the concrete syntax and semantics of the language in
the following sections.}

\section{Hello, World! in STL Script}
\begin{verbatim}
    println "Hello, World!"
\end{verbatim}

\noindent{\texttt{println} is a built-in function in STL Script. It will print whatever expression is behind and append a \verb|\n| character at the end of the expression to create a newline afterwards. \texttt{"Hello, World!"} is a string, which is one of the primitive types in the language.}

\section{Preliminaries}

\subsection{Primitive Types}

\noindent{STL Script is a statically-typed language, it supports both implicit or explicit typing, and it has the following primitive types built into the language:}
\begin{itemize}
    \item Int
    \item Float
    \item String
    \item Boolean
    \item List
    \item Tuple
    \item STL
\end{itemize}

\noindent{Most of the primitive types are similar to those in programming languages like Scala or Java. Note that \texttt{List} is a parametric type. It can accept another primitive
type as an argument. For example, \texttt{List<Int>} indicates a \texttt{List}
consists of Integers (of \texttt{Int} Type). \texttt{STL} type is assigned to
STL formulas.}

\subsection{Values, Expressions and Statements}

\subsubsection{Values}

\noindent{Values are language components that cannot be evaluated any further.
They are the building block for Expressions. Instances of Values can be \texttt{3.1415926}, \texttt{42}, $\cdots$}

\subsubsection{Expressions}

\noindent{Expressions are all language components that can be evaluated to a
value, and does not have any side effects (i.e. assignment to a variable, or
write to standard output or file streams, etc). They are the building block for Statements.}

\medskip

\noindent{Instances of Expressions can be \texttt{1+1}, \texttt{true||false}, $\cdots$}

\subsubsection{Statements}

\noindent{Statements are all language components that perform certain actions,
and may exhibit side effects. They typically consist of Expressions, and they are followed by separators like  \verb|;| or \verb|\n| (or both) characters.}

\medskip

\noindent{Instances of Statements can be \texttt{val i = 3;} \quad \texttt{println "Hello, World!"; }}

\subsection{Function Invocation}
\noindent{Function can be invoked on expressions. The invocation is initiated by \textbf{.} (dot). For example, in the following assignment statement of STL formula:}
\begin{verbatim}
    val property = G[0,1]($distance_to_boundary > 5.0)(0, signal)
\end{verbatim}

\noindent{Note that the signal in our case can be any arbitrary signal (we will discuss this later). We can evaluate the satisfaction of the STL formula by invoking the \texttt{eval()} function on the property variable like the following:}

\begin{verbatim}
    property.eval()
\end{verbatim}

\noindent{This will evaluate the STL formula to a \texttt{Boolean} value \texttt{true} or \texttt{false}. The invocation itself is a expression.}

\section{STL Specification}
\end{document}
